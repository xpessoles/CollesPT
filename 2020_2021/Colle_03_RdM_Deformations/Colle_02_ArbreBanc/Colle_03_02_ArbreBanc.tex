\documentclass[10pt,fleqn]{article} % Default font size and left-justified equations
\usepackage[%
    pdftitle={X Pessoles RDM },
    pdfauthor={Xavier Pessoles}]{hyperref}

    
\input{style/new_style}
\input{style/macros_SII}
\usepackage{multicol}
\usepackage{siunitx}
%\usepackage{picins}
\fichetrue
%\fichefalse

\proftrue

\proffalse

\tdtrue
%\tdfalse

\courstrue
\coursfalse


\def\classe{\textsf{PT}}
\def\xxnumpartie{}%Cycle --}
\def\xxpartie{ }

\def\xxnumchapitre{}%Chapitre -- \vspace{.2cm}}
\def\xxchapitre{\hspace{.12cm} }

\def\discipline{Sciences \\Industrielles de \\ l'Ingénieur}
\def\xxtete{Sciences Industrielles de l'Ingénieur}


  
\def\xxposongletx{2}
\def\xxposonglettext{1.45}
\def\xxposonglety{20}
%\def\xxonglet{Part. 1 -- Ch. 3}
\def\xxonglet{\textsf{}}%Cycle 05}}

\def\xxactivite{Colle 01}
\def\xxauteur{\textsl{Xavier Pessoles}}


\def\xxtitreexo{Banc d'essai de boîte de transmission principale (BTP)}
\def\xxsourceexo{\hspace{.2cm} \footnotesize{CCP TSI}}


\def\xxcompetences{%
\vspace{-.5cm}
\footnotesize{
\textsl{%
\textbf{Savoirs et compétences :}\\
\vspace{-.2cm}
%\begin{itemize}[label=\ding{112},font=\color{ocre}] 
%\item Mod2.C18.SF1 : Déterminer l’énergie cinétique d’un solide, ou d’un ensemble de solides, dans son mouvement par rapport à un autre solide.
%\item Res1.C1.SF1 : Proposer une démarche permettant la détermination de la loi de mouvement.
%\item Mod1.C5.SF2 : Déterminer la puissance des actions mécaniques extérieures à un solide ou à un ensemble de solides, dans son mouvement rapport à un autre solide.
%\item Mod1.C5.SF3 : Déterminer la puissance des actions mécaniques intérieures à un ensemble de solides.
%\end{itemize}
}}}

\def\xxfigures{
\includegraphics[width=.8\textwidth]{images/fig_01}
}%figues de la page de garde


\def\xxpied{%
%Cycle 05 -- Modélisation mécanique -- Énergétique\\% afin de valider leurs performances.\\
%Chapitre 1 -- \xxactivite%
}

\setcounter{secnumdepth}{5}
%---------------------------------------------------------------------------


\begin{document}
%\chapterimage{png/Fond_Cin}
\input{style/new_pagegarde}
\vspace{4.5cm}
\pagestyle{fancy}
\thispagestyle{plain}


\def\columnseprulecolor{\color{ocre}}
\setlength{\columnseprule}{0.4pt} 

%\ifprof
%\else
\begin{multicols}{2}
%\fi
\section*{Mise en situation}



%Airbus Helicopters commercialise des hélicoptères civils et militaires. Pour les États ou les entreprises faisant l’acquisition de ces machines, un des critères de choix est la masse qui
%peut être embarquée ou déplacée. Ainsi, pour
%les hélicoptères de la gamme EC 145, la masse
%à transporter est de 3585 kg. Cette charge va
%influer sur les dimensions de l’appareil et sur
%la puissance à fournir par les turboréacteurs.
%Le déplacement des hélicoptères est assuré
%par un rotor principal permettant la sustentation
%et la translation de l’appareil. Un rotor
%arrière permet de compenser le couple de réaction engendré par le rotor principal et
%de contrôler les mouvements de lacet de l’appareil. La puissance est délivrée par deux
%turboréacteurs (certains hélicoptères ne sont équipés que d’un turboréacteur). Ces
%turboréacteurs entraînent en rotation une boîte de transmission principale.
La Boite de Transmission Principale (BTP) permet de distribuer la puissance au
rotor principal, au rotor de queue ainsi qu’à différents accessoires d'un hélicoptère (alternateur, pompe
hydraulique, etc.).



Afin d’évaluer la qualité de la BTP, un banc d’essai permet de la
solliciter et de recréer les conditions de vol. Le diagramme des exigences partiel des exigences est donné figure suivante.


\begin{center}
\includegraphics[width=\linewidth]{images/fig_02}
\end{center}
%
%Le banc d’essai se présente sous la forme d’un châssis permettant d’assurer la liaison avec la BTP. Il est équipé d’un moteur à courant continu piloté par un variateur lui-même alimenté par un transformateur. Ce moteur
%entraîne une succession de réducteurs entraînant eux-mêmes deux arbres reliés aux deux
%entrées de la BTP. La BTP agit alors sur le rotor principal de l’hélicoptère. Le respect des
%caractéristiques des cycles d’essais est assuré par un asservissement en vitesse et en couple.
%Des vérins (intégrés au système de simulation des efforts) permettent d’appliquer les efforts simulant la portance de l’appareil en sortie de la BTP.

\begin{obj}
L’objectif de cette étude est de dimensionner l’arbre faisant la jonction entre la
BTP et le banc d’essai.
\end{obj}

\begin{center}
\includegraphics[width=.8\linewidth]{images/fig_03}
\end{center}






\subsection*{Modélisation de la structure}

%Dans le cadre d’un essai de la BTP, les pales ne sont pas utilisées. Il est donc nécessaire
%de concevoir un arbre de sortie qui doit faire office de rotor principal. Cet arbre de sortie
%devra supporter des efforts équivalents à ceux engendrés par les efforts aérodynamiques.
%L’accouplement avec le reste du banc d’essai doit permettre de fournir un couple
%résistant. Par ailleurs, des vérins permettent de générer l’équivalent de
%l’effort de portance. Une modélisation de l’arbre de sortie de la BTP est présentée sur la
%figure  suivante.

Le modèle suivant présente l'arbre de sortie de BTP intégré dans le banc d'essai.

\begin{center}
\includegraphics[width=\linewidth]{images/fig_04}
\end{center}

On étudie la phase de vie où l’hélicoptère passe d’une condition de vol stationnaire
à un déplacement. Cette configuration du banc d’essai se traduit par la modélisation
donnée figure suivante. Les efforts sont les suivants en $A$, $B$ et $C$ dans le repère $\repere{O}{x_S}{y_s}{z_s}$.
$
\torseurstat{T}{\text{Ext}_A}{1} = \torseurcol{0}{F_y}{0}{0}{0}{0}{A,\mathcal{R}_S}, 
\torseurstat{T}{\text{Ext}_B}{1} = \torseurcol{-F_x}{-F_y}{0}{0}{0}{0}{B,\mathcal{R}_S}, 
\torseurstat{T}{\text{Ext}_C}{1} = \torseurcol{F_x}{0}{0}{C_t}{0}{0}{C,\mathcal{R}_S}
$.

\begin{center}
\includegraphics[width=\linewidth]{images/fig_05}
\end{center}

\begin{multicols}{2}
\begin{itemize}
\item $OA = \ell_1 = \SI{100}{mm}$;
\item $OB = \ell_2 = \SI{200}{mm}$;
\item $OC = \ell_3 = \SI{300}{mm}$;
\item $F_x = \SI{80000}{N}$; 
\item $F_y = \SI{120000}{N}$;
\item $Ct = \SI{4100}{Nm}$;
\item $R_e = \SI{600}{Mpa}$ 
\item limite d’élasticité au glissement : $R_{eg} = 0,5R_e$.
\end{itemize}
\end{multicols}


\subsection*{Évaluer les contraintes dans l’arbre}

\subparagraph{}\textit{Après avoir identifié les différents tronçons à étudier, déterminer le torseur de
cohésion dans l’arbre de sortie de la BTP.}

\subparagraph{}\textit{Tracer les diagrammes des sollicitations associés à chacune des composantes du
torseur de cohésion.}

\subparagraph{}\textit{En ne tenant compte que des sollicitation donnant lieu à une contrainte normale,
donner, dans chacun des tronçons :
\begin{itemize}
\item l’expression vectorielle de contrainte normale;
\item l’allure du champ des contraintes;
\item le diamètre minimal de l’arbre permettant de rester dans le domaine élastique.
\end{itemize}}





\subparagraph{}\textit{En ne prenant en compte que le moment de torsion, déterminer quel serait le rayon minimal de la poutre. }

\subparagraph{}\textit{Sur le tronçon $]AB[$, déterminer l'expression de la norme du vecteur << contrainte de cisaillement >>.  Tracer 
		l'allure du champ des contraintes sur ce tronçon.
		Pour réaliser ce calcul, on pourra introduire le système de coordonnées polaires $(\rho,\theta)$ décrit
		par la figure ci-dessous.}

\begin{center}
\includegraphics[width=.4\linewidth]{images/fig_06}
\end{center}

		Il n'est pas possible de sommer des contraintes normales et des contraintes de cisaillement. On utilise 
alors le critère de von Mises, basé sur l'utilisation d'une contrainte équivalente. En utilisant le système de coordonnées polaires dans une section droite, on définit cette contrainte équivalente par 
$\sigma_{vm}(\rho,\theta)=\sqrt{\sigma_n^2(\rho,\theta)+3\tau^2(\rho,\theta)}$ où $\sigma_n$ et $\tau$ représentent 
respectivement les contraintes normale et tangentielle. Le critère de von Mises est alors : $\sigma_{\text{vm}} \leq R_e$.


\subparagraph{}\textit{Exprimer $\sigma_{\text{vm}}(\rho, \theta)$ pour tout point d'une section circulaire de diamètre $D$ du tronçon $]AB[$. On exprimera le résultat en fonction de $C_t$, $F_y$, $D$, $\lambda$, $\ell_2$.}
\subparagraph{}\textit{ Pour quelles valeurs de $\lambda$ et $\rho$ cette contrainte est-elle maximale ? Exprimer alors le critère de dimensionnement à appliquer en fonction de $D$, $F_y$, 
$C_t$, $\ell_1$, $\ell_2$, $\theta$ et $R_e$.}


La figure  suivante  représente la fonction ${R_e}-\sigma_{\text{vm}}$ pour $\theta$ compris dans l'intervalle $\left[-\pi,\pi\right]$ pour différents diamètres.

\begin{center}
\includegraphics[width=\linewidth]{images/courbes}
\end{center}

\subparagraph{}\textit{On choisit un diamètre de \SI{59}{mm} pour le tronçon $]BC[$. Proposer un diamètre de l'arbre pour le tronçon $]AB[$ en utilisant la figure précédente.}



\subparagraph{}\textit{En conservant les dimensions précédentes et en considérant une phase de vie où l'arbre n'est soumis qu'à de la torsion, déterminer, en utilisant l'abaque de concentration de contraintes en torsion de la figure suivante le rayon de raccordement pour que le coefficient de concentration de contrainte soit inférieur à 1,5.}



\begin{center}
\includegraphics[width=\linewidth]{images/concentration_contrainte_torsion}
\end{center}



\begin{center}
\includegraphics[width=.6\linewidth]{images/fig_06_bis}
\end{center}

\subsection*{Conclusion: Retour par rapport à l'objectif initial}

Une analyse éléments finis a été menée sur l'arbre prédimensionné par la démarche que nous venons de suivre (voir Figure ci-dessous).

\begin{center}
\includegraphics[width=.4\linewidth]{images/EF}
\end{center}


\subparagraph{}\textit{Que peut-on en conclure ? Justifiez le besoin d'utiliser un coefficient de sécurité pour mener les études de prédimensionnement. 
Dans le cas où un coefficient de sécurité serait imposé dans un nouveau cahier des charges, quelles seraient les étapes à modifier?}



%\subparagraph{}\textit{}

%\subsection*{Flexion de l’arbre intermédiaire 2}
%\begin{hypo}
%Dans cette étude, le candidat ne prendra en compte que le moment de flexion $Mf_y$.
%\end{hypo}

\end{multicols}



\newpage
\begin{center}
\includegraphics[width=\linewidth]{images/cor_01}
\end{center}

\begin{center}
\includegraphics[width=\linewidth]{images/cor_02}

\includegraphics[width=\linewidth]{images/cor_03}

\includegraphics[width=\linewidth]{images/cor_04}
\end{center}

\end{document}

\subparagraph{}\textit{}
\ifprof
\begin{corrige}~\\
\end{corrige}
\else
\fi




\subparagraph{}\textit{}
\ifprof
\begin{corrige}~\\
\end{corrige}
\else
\fi

\subparagraph{}\textit{}
\ifprof
\begin{corrige}~\\
\end{corrige}
\else
\fi

\subparagraph{}\textit{}
\ifprof
\begin{corrige}~\\
\end{corrige}
\else
\fi

\subparagraph{}\textit{}
\ifprof
\begin{corrige}~\\
\end{corrige}
\else
\fi

\subparagraph{}\textit{}
\ifprof
\begin{corrige}~\\
\end{corrige}
\else
\fi

\subparagraph{}\textit{}
\ifprof
\begin{corrige}~\\
\end{corrige}
\else
\fi
\begin{center}
%\includegraphics[width=\linewidth]{images/fig_05}
\end{center}
