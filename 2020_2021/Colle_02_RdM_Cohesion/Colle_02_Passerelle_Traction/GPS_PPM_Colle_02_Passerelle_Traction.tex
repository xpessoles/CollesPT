\documentclass[10pt,fleqn]{article} % Default font size and left-justified equations
\usepackage[%
    pdftitle={Energétique},
    pdfauthor={Xavier Pessoles}]{hyperref}

    
\input{style/new_style}
\input{style/macros_SII}
\usepackage{multicol}
\usepackage{siunitx}
%\usepackage{picins}
\fichetrue
%\fichefalse

\proftrue

\proffalse

\tdtrue
%\tdfalse

\courstrue
\coursfalse


\def\classe{\textsf{PT}}
\def\xxnumpartie{}%Cycle --}
\def\xxpartie{ }

\def\xxnumchapitre{}%Chapitre -- \vspace{.2cm}}
\def\xxchapitre{\hspace{.12cm} }

\def\discipline{Sciences \\Industrielles de \\ l'Ingénieur}
\def\xxtete{Sciences Industrielles de l'Ingénieur}


  
\def\xxposongletx{2}
\def\xxposonglettext{1.45}
\def\xxposonglety{20}
%\def\xxonglet{Part. 1 -- Ch. 3}
\def\xxonglet{\textsf{}}%Cycle 05}}

\def\xxactivite{Colle 02}
\def\xxauteur{\textsl{Xavier Pessoles}}


\def\xxtitreexo{Modèle de passerelle}
\def\xxsourceexo{\hspace{.2cm} \footnotesize{Xavier Pessoles}}


\def\xxcompetences{%
\vspace{-.5cm}
\footnotesize{
\textsl{%
\textbf{Savoirs et compétences :}\\
\vspace{-.2cm}
%\begin{itemize}[label=\ding{112},font=\color{ocre}] 
%\item Mod2.C18.SF1 : Déterminer l’énergie cinétique d’un solide, ou d’un ensemble de solides, dans son mouvement par rapport à un autre solide.
%\item Res1.C1.SF1 : Proposer une démarche permettant la détermination de la loi de mouvement.
%\item Mod1.C5.SF2 : Déterminer la puissance des actions mécaniques extérieures à un solide ou à un ensemble de solides, dans son mouvement rapport à un autre solide.
%\item Mod1.C5.SF3 : Déterminer la puissance des actions mécaniques intérieures à un ensemble de solides.
%\end{itemize}
}}}

\def\xxfigures{
\includegraphics[width=.55\textwidth]{images/fig_01}
}%figues de la page de garde


\def\xxpied{%
%Cycle 05 -- Modélisation mécanique -- Énergétique\\% afin de valider leurs performances.\\
%Chapitre 1 -- \xxactivite%
}

\setcounter{secnumdepth}{5}
%---------------------------------------------------------------------------


\begin{document}
%\chapterimage{png/Fond_Cin}
\input{style/new_pagegarde}
\vspace{4.5cm}
\pagestyle{fancy}
\thispagestyle{plain}


\def\columnseprulecolor{\color{ocre}}
\setlength{\columnseprule}{0.4pt} 

%\ifprof
%\else
\begin{multicols}{2}
%\fi
\section*{Mise en situation}


\begin{center}
\includegraphics[width=.7\linewidth]{images/fig_02}

\textit{Passerelle réelle}
\end{center}

\begin{center}
\includegraphics[width=.8\linewidth]{images/fig_03}

\textit{Modèle choisi}
\end{center}

On s'intéresse au dimensionnement des haubans \textbf{(2)} permettant de maintenir en équilibre une passerelle.
On modélise la charge sur le pont comme une charge linéique~$c$.

\subsection*{Détermination du torseur de cohésion}
\subparagraph{}\textit{Réaliser le paramétrage du problème.}
\ifprof
\begin{corrige}~\\
\end{corrige}
\else
\fi

\subparagraph{}\textit{Déterminer les actions mécaniques dans les liaisons.}
\ifprof
\begin{corrige}~\\
\end{corrige}
\else
\fi

\subparagraph{}\textit{Déterminer le torseur de cohésion dans les poutres \textbf{(1)} et \textbf{(2)}.}
\ifprof
\begin{corrige}~\\
\end{corrige}
\else
\fi

\subparagraph{}\textit{Tracer les diagrammes des sollicitations.}
\ifprof
\begin{corrige}~\\
\end{corrige}
\else
\fi


\subsection*{Déformation du hauban et déplacement de la structure}
On considère ici que le pont (1) est indéformable, mais que le hauban (2) est déformable. 
\subparagraph{}\textit{Déterminer l'allongement du câble.}
\ifprof
\begin{corrige}~\\
\end{corrige}
\else
\fi

\subparagraph{}\textit{En faisant l'hypothèse que la rotation de la passerelle en $A$ est << petite >>, déterminer le déplacement du point $B$ puis du point $C$. }
\ifprof
\begin{corrige}~\\
\end{corrige}
\else
\fi

\subsection*{Moment quadratique}
La section de la passerelle est donnée figure suivante. 
\begin{center}
\includegraphics[width=.65\linewidth]{images/fig_04}

%\textit{Modèle choisi}
\end{center}

\subparagraph{}\textit{Déterminer le moment quadratique en $O$ par rapport à $\vect{y}$ puis par rapport à $\vect{z}$. }


\end{multicols}

\end{document}

\subparagraph{}\textit{}
\ifprof
\begin{corrige}~\\
\end{corrige}
\else
\fi




\subparagraph{}\textit{}
\ifprof
\begin{corrige}~\\
\end{corrige}
\else
\fi

\subparagraph{}\textit{}
\ifprof
\begin{corrige}~\\
\end{corrige}
\else
\fi

\subparagraph{}\textit{}
\ifprof
\begin{corrige}~\\
\end{corrige}
\else
\fi

\subparagraph{}\textit{}
\ifprof
\begin{corrige}~\\
\end{corrige}
\else
\fi

\subparagraph{}\textit{}
\ifprof
\begin{corrige}~\\
\end{corrige}
\else
\fi

\subparagraph{}\textit{}
\ifprof
\begin{corrige}~\\
\end{corrige}
\else
\fi
\begin{center}
%\includegraphics[width=\linewidth]{images/fig_05}
\end{center}
