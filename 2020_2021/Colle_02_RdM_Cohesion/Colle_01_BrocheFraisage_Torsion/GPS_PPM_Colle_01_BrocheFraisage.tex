\documentclass[10pt,fleqn]{article} % Default font size and left-justified equations
\usepackage[%
    pdftitle={Energétique},
    pdfauthor={Xavier Pessoles}]{hyperref}

    
\input{style/new_style}
\input{style/macros_SII}
\usepackage{multicol}
\usepackage{siunitx}
%\usepackage{picins}
\fichetrue
%\fichefalse

\proftrue

\proffalse

\tdtrue
%\tdfalse

\courstrue
\coursfalse


\def\classe{\textsf{PT}}
\def\xxnumpartie{}%Cycle --}
\def\xxpartie{ }

\def\xxnumchapitre{}%Chapitre -- \vspace{.2cm}}
\def\xxchapitre{\hspace{.12cm} }

\def\discipline{Sciences \\Industrielles de \\ l'Ingénieur}
\def\xxtete{Sciences Industrielles de l'Ingénieur}


  
\def\xxposongletx{2}
\def\xxposonglettext{1.45}
\def\xxposonglety{20}
%\def\xxonglet{Part. 1 -- Ch. 3}
\def\xxonglet{\textsf{}}%Cycle 05}}

\def\xxactivite{Colle 01}
\def\xxauteur{\textsl{Xavier Pessoles}}


\def\xxtitreexo{Système d’entraînement d'une broche de fraisage}
\def\xxsourceexo{\hspace{.2cm} \footnotesize{SIB 2007}}


\def\xxcompetences{%
\vspace{-.5cm}
\footnotesize{
\textsl{%
\textbf{Savoirs et compétences :}\\
\vspace{-.2cm}
%\begin{itemize}[label=\ding{112},font=\color{ocre}] 
%\item Mod2.C18.SF1 : Déterminer l’énergie cinétique d’un solide, ou d’un ensemble de solides, dans son mouvement par rapport à un autre solide.
%\item Res1.C1.SF1 : Proposer une démarche permettant la détermination de la loi de mouvement.
%\item Mod1.C5.SF2 : Déterminer la puissance des actions mécaniques extérieures à un solide ou à un ensemble de solides, dans son mouvement rapport à un autre solide.
%\item Mod1.C5.SF3 : Déterminer la puissance des actions mécaniques intérieures à un ensemble de solides.
%\end{itemize}
}}}

\def\xxfigures{
\includegraphics[width=.55\textwidth]{images/fig_01}
}%figues de la page de garde


\def\xxpied{%
%Cycle 05 -- Modélisation mécanique -- Énergétique\\% afin de valider leurs performances.\\
%Chapitre 1 -- \xxactivite%
}

\setcounter{secnumdepth}{5}
%---------------------------------------------------------------------------


\begin{document}
%\chapterimage{png/Fond_Cin}
\input{style/new_pagegarde}
\vspace{4.5cm}
\pagestyle{fancy}
\thispagestyle{plain}


\def\columnseprulecolor{\color{ocre}}
\setlength{\columnseprule}{0.4pt} 

%\ifprof
%\else
\begin{multicols}{2}
%\fi
\section*{Mise en situation}
La figure au verso illustre la cinématique permettant la rotation d'une broche de fraisage sur un centre d'usinage multiaxes.
On s'intéresse en particulier à l'arbre intermédiaire 2. Celui-ci est modélisé  par une poutre de diamètre $D$ et de longueur utile $L$. Les variations de diamètres seront négligées.

\begin{center}
\includegraphics[width=\linewidth]{images/fig_03}
\end{center}

Les points $A$ et $B$ sont les centres d’inertie géométriques des sections droites contenant
respectivement les points $P$ et $S$.
En considérant la composante $A_{32}$ des efforts de la roue sur la vis dans le sens $\vect{x}$ positif, les
torseurs des actions mécaniques extérieures qui s’exercent sur l’arbre intermédiaire 2, dans la base
$\base{x}{y}{z}$, sont :
\footnotesize
$\torseurstat{T}{4}{2}_1 = \torseurcol{X_{Q_1}}{Y_{Q_1}}{Z_{Q_1}}{0}{0}{0}{Q_1}$, 
$\torseurstat{T}{4}{2}_2 = \torseurcol{X_{Q_2}}{Y_{Q_2}}{Z_{Q_2}}{0}{0}{0}{Q_2}$, 
$\torseurstat{T}{3}{2} = \torseurcol{A_{32}}{-R_{32}}{T_{32}}{0}{0}{0}{P}$, 
$\torseurstat{T}{1}{2} = \torseurcol{0}{-R_{12}}{-T_{12}}{0}{0}{0}{S}$.

\normalsize 

\subparagraph{}\textit{Proposer une méthode permettant de déterminer l'expression du torseur des efforts intérieurs au centre d'inertie de chaque section droite.}

\subparagraph{}\textit{Mettre en \oe{}uvre cette méthode pour déterminer le torseur de cohésion.}

\subparagraph{}\textit{Tracer les diagrammes des sollicitations en fonction de l'abscisse du centre d'inertie de la section droite.}

\subsection*{Torsion de l’arbre intermédiaire 2}

Le module de Coulomb du matériau utilisé est : $G = \SI{80000}{MPa}$.

\subparagraph{}\textit{Déterminer l’expression, en fonction de $T_{12}$, $d_2$, $G$, $L_2$, $L_3$ et $\theta_{\text{lim}}$, du diamètre minimum $D_{\text{min}}$ de l’arbre 2, pour que le déphasage $\theta$ des sections passant par le point
$P$ et par le point $S$ soit inférieur à la valeur limite $\theta_{\text{lim}}$.}

\subparagraph{}\textit{Dans le système étudié, le constructeur souhaite $\theta_{\text{lim}}=0,1\degres$. Donner la valeur
numérique de $D_{\text{min}}$.}

\subparagraph{}\textit{Du fait de l’existence de ce déphasage de sections et vis-à-vis du système étudié, quelle
est le meilleur emplacement pour positionner le capteur de position. Doit-on le positionner
sur le moteur ou sur la broche elle-même ? Justifier brièvement votre réponse.}

%\subsection*{Flexion de l’arbre intermédiaire 2}
%\begin{hypo}
%Dans cette étude, le candidat ne prendra en compte que le moment de flexion $Mf_y$.
%\end{hypo}

\end{multicols}


\begin{center}
\includegraphics[width=\linewidth]{images/fig_02}
\end{center}

\newpage
\begin{center}
\includegraphics[width=\linewidth]{images/cor_01}
\end{center}

\begin{center}
\includegraphics[width=\linewidth]{images/cor_02}
\end{center}

\end{document}

\subparagraph{}\textit{}
\ifprof
\begin{corrige}~\\
\end{corrige}
\else
\fi




\subparagraph{}\textit{}
\ifprof
\begin{corrige}~\\
\end{corrige}
\else
\fi

\subparagraph{}\textit{}
\ifprof
\begin{corrige}~\\
\end{corrige}
\else
\fi

\subparagraph{}\textit{}
\ifprof
\begin{corrige}~\\
\end{corrige}
\else
\fi

\subparagraph{}\textit{}
\ifprof
\begin{corrige}~\\
\end{corrige}
\else
\fi

\subparagraph{}\textit{}
\ifprof
\begin{corrige}~\\
\end{corrige}
\else
\fi

\subparagraph{}\textit{}
\ifprof
\begin{corrige}~\\
\end{corrige}
\else
\fi
\begin{center}
%\includegraphics[width=\linewidth]{images/fig_05}
\end{center}
