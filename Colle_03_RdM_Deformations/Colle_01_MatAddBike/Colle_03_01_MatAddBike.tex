\documentclass[10pt,fleqn]{article} % Default font size and left-justified equations
\usepackage[%
    pdftitle={X Pessoles RDM },
    pdfauthor={Xavier Pessoles}]{hyperref}

    
\input{style/new_style}
\input{style/macros_SII}
\usepackage{multicol}
\usepackage{siunitx}
%\usepackage{picins}
\fichetrue
%\fichefalse

\proftrue

\proffalse

\tdtrue
%\tdfalse

\courstrue
\coursfalse


\def\classe{\textsf{PT}}
\def\xxnumpartie{}%Cycle --}
\def\xxpartie{ }

\def\xxnumchapitre{}%Chapitre -- \vspace{.2cm}}
\def\xxchapitre{\hspace{.12cm} }

\def\discipline{Sciences \\Industrielles de \\ l'Ingénieur}
\def\xxtete{Sciences Industrielles de l'Ingénieur}


  
\def\xxposongletx{2}
\def\xxposonglettext{1.45}
\def\xxposonglety{20}
%\def\xxonglet{Part. 1 -- Ch. 3}
\def\xxonglet{\textsf{}}%Cycle 05}}

\def\xxactivite{Colle 01}
\def\xxauteur{\textsl{Xavier Pessoles}}


\def\xxtitreexo{Mât de l'AddBike}
\def\xxsourceexo{\hspace{.2cm} \footnotesize{Agrégation SII -- 2018}}


\def\xxcompetences{%
\vspace{-.5cm}
\footnotesize{
\textsl{%
\textbf{Savoirs et compétences :}\\
\vspace{-.2cm}
%\begin{itemize}[label=\ding{112},font=\color{ocre}] 
%\item Mod2.C18.SF1 : Déterminer l’énergie cinétique d’un solide, ou d’un ensemble de solides, dans son mouvement par rapport à un autre solide.
%\item Res1.C1.SF1 : Proposer une démarche permettant la détermination de la loi de mouvement.
%\item Mod1.C5.SF2 : Déterminer la puissance des actions mécaniques extérieures à un solide ou à un ensemble de solides, dans son mouvement rapport à un autre solide.
%\item Mod1.C5.SF3 : Déterminer la puissance des actions mécaniques intérieures à un ensemble de solides.
%\end{itemize}
}}}

\def\xxfigures{
\includegraphics[width=.55\textwidth]{images/fig_01}
}%figues de la page de garde


\def\xxpied{%
%Cycle 05 -- Modélisation mécanique -- Énergétique\\% afin de valider leurs performances.\\
%Chapitre 1 -- \xxactivite%
}

\setcounter{secnumdepth}{5}
%---------------------------------------------------------------------------


\begin{document}
%\chapterimage{png/Fond_Cin}
\input{style/new_pagegarde}
\vspace{4.5cm}
\pagestyle{fancy}
\thispagestyle{plain}


\def\columnseprulecolor{\color{ocre}}
\setlength{\columnseprule}{0.4pt} 

%\ifprof
%\else
\begin{multicols}{2}
%\fi
\section*{Mise en situation}
Le dispositif Bi-roue est un système développé par une start-up dont le but est de développer l’utilisation d’un vélo afin d’en faire une réelle alternative aux autres moyens de transport. Ce produit doit s’adapter à tous types de vélo et doit permettre de transporter de la marchandise (colis ou courses du quotidien) ou encore des enfants.

On représente ci-dessous le système sans les roues et sans le système de transport. 

\begin{center}
\includegraphics[width=\linewidth]{images/fig_02}
\end{center}


Un extrait des exigences est donné ci-dessous.
\begin{center}
\includegraphics[width=\linewidth]{images/fig_06}
\end{center}



\subsection*{Modélisation retenue}
On s'intéresse au dimensionnement du mât. Cet élément participe à la rigidité de la liaison entre la fourche du vélo et l'AddBike. Le modèle retenu est proposé figure suivante. 


\begin{center}
\includegraphics[width=\linewidth]{images/fig_03}
\end{center}


\subparagraph{}\textit{Déterminer le torseur de cohésion dans chacun des deux tronçons de la poutre.} 
\subparagraph{}\textit{Tracer les diagrammes des sollicitations et préciser leur nom.}
\subsection*{Choix du matériau et de la géométrie}

Quels que soient les résultats de la partie précédente, on considère que le mât n’est soumis qu’à de la flexion. On néglige l’angle du mât. Dans le but de dimensionner la section du mât, on cherche à connaître le matériau proposant le meilleur compromis masse -- tenue sans déformation permanente. 
Les paramètres $a$ et $d$  sont appelés paramètres fixes et $c$  le paramètre ajustable. 
On note $\rho$ la masse volumique du matériau. 

\subparagraph{}\textit{Donner l’expression de la contrainte normale maximale dans le mât. 
Exprimer sa masse en fonction des paramètres fixes ($a$  et $d$) , du moment de flexion, de la masse volumique et de la contrainte normale. }
	

On admet que la masse est proportionnelle à $\dfrac{\rho}{\sigma_{\text{maxi}}^{2/3}}$ .

\subparagraph{}\textit{Quel critère faut-il maximiser pour minimiser la masse du mât tout en garantissant un fonctionnement sans déformation permanente de la pièce ?}
\subparagraph{}\textit{En utilisant la carte d’Ashby (limite élastique en fonction de la masse volumique), proposer des matériaux permettant d’obtenir les meilleurs compromis masse – contrainte élastique.}
	

On considère que le moment maximal est de \SI{150}{N.m}. 
\subparagraph{}\textit{En utilisant un coefficient de sécurité de 2 et les résultats des questions précédentes, choisir un matériau et déterminer la valeur de $c$  permettant que l’exigence 1.4.2 soit satisfaite.}


\subsection*{Détermination des déformations}

\subparagraph{}\textit{Proposer une méthode permettant de déterminer le déplacement du point $P$.}

\subparagraph{}\textit{Après validation du professeur, mettre en \oe{}uvre cette méthode. .}
\end{multicols}


\begin{center}
\includegraphics[width=\linewidth]{images/fig_07}
\end{center}

\newpage
\begin{center}
\includegraphics[width=\linewidth]{images/cor_01}
\includegraphics[width=\linewidth]{images/cor_02}
\includegraphics[width=\linewidth]{images/cor_03}
\includegraphics[width=\linewidth]{images/cor_04}
\includegraphics[width=\linewidth]{images/cor_05}
\includegraphics[width=\linewidth]{images/cor_06}
\end{center}

\begin{center}

\end{center}

\end{document}

\subparagraph{}\textit{}
\ifprof
\begin{corrige}~\\
\end{corrige}
\else
\fi




\subparagraph{}\textit{}
\ifprof
\begin{corrige}~\\
\end{corrige}
\else
\fi

\subparagraph{}\textit{}
\ifprof
\begin{corrige}~\\
\end{corrige}
\else
\fi

\subparagraph{}\textit{}
\ifprof
\begin{corrige}~\\
\end{corrige}
\else
\fi

\subparagraph{}\textit{}
\ifprof
\begin{corrige}~\\
\end{corrige}
\else
\fi

\subparagraph{}\textit{}
\ifprof
\begin{corrige}~\\
\end{corrige}
\else
\fi

\subparagraph{}\textit{}
\ifprof
\begin{corrige}~\\
\end{corrige}
\else
\fi
\begin{center}
%\includegraphics[width=\linewidth]{images/fig_05}
\end{center}
