\documentclass[10pt,fleqn]{article} % Default font size and left-justified equations
\usepackage[%
    pdftitle={X Pessoles RDM },
    pdfauthor={Xavier Pessoles}]{hyperref}

    
\input{style/new_style}
\input{style/macros_SII}
\usepackage{multicol}
\usepackage{siunitx}
%\usepackage{picins}
\fichetrue
%\fichefalse

\proftrue

\proffalse

\tdtrue
%\tdfalse

\courstrue
\coursfalse


\def\classe{\textsf{PT}}
\def\xxnumpartie{}%Cycle --}
\def\xxpartie{ }

\def\xxnumchapitre{}%Chapitre -- \vspace{.2cm}}
\def\xxchapitre{\hspace{.12cm} }

\def\discipline{Sciences \\Industrielles de \\ l'Ingénieur}
\def\xxtete{Sciences Industrielles de l'Ingénieur}


  
\def\xxposongletx{2}
\def\xxposonglettext{1.45}
\def\xxposonglety{20}
%\def\xxonglet{Part. 1 -- Ch. 3}
\def\xxonglet{\textsf{}}%Cycle 05}}

\def\xxactivite{Colle 01}
\def\xxauteur{\textsl{Xavier Pessoles}}


\def\xxtitreexo{Banc d'essai de boîte de transfert}
\def\xxsourceexo{\hspace{.2cm} \footnotesize{CCP TSI}}


\def\xxcompetences{%
\vspace{-.5cm}
\footnotesize{
\textsl{%
\textbf{Savoirs et compétences :}\\
\vspace{-.2cm}
%\begin{itemize}[label=\ding{112},font=\color{ocre}] 
%\item Mod2.C18.SF1 : Déterminer l’énergie cinétique d’un solide, ou d’un ensemble de solides, dans son mouvement par rapport à un autre solide.
%\item Res1.C1.SF1 : Proposer une démarche permettant la détermination de la loi de mouvement.
%\item Mod1.C5.SF2 : Déterminer la puissance des actions mécaniques extérieures à un solide ou à un ensemble de solides, dans son mouvement rapport à un autre solide.
%\item Mod1.C5.SF3 : Déterminer la puissance des actions mécaniques intérieures à un ensemble de solides.
%\end{itemize}
}}}

\def\xxfigures{
\includegraphics[width=.8\textwidth]{images/fig_01}
}%figues de la page de garde


\def\xxpied{%
%Cycle 05 -- Modélisation mécanique -- Énergétique\\% afin de valider leurs performances.\\
%Chapitre 1 -- \xxactivite%
}

\setcounter{secnumdepth}{5}
%---------------------------------------------------------------------------


\begin{document}
%\chapterimage{png/Fond_Cin}
\input{style/new_pagegarde}
\vspace{4.5cm}
\pagestyle{fancy}
\thispagestyle{plain}


\def\columnseprulecolor{\color{ocre}}
\setlength{\columnseprule}{0.4pt} 

%\ifprof
%\else
\begin{multicols}{2}
%\fi
\section*{Mise en situation}

$\base{x}{y}{z}$, sont :
\footnotesize
Airbus Helicopters commercialise des hélicoptères civils et militaires. Pour les États ou les entreprises faisant l’acquisition de ces machines, un des critères de choix est la masse qui
peut être embarquée ou déplacée. Ainsi, pour
les hélicoptères de la gamme EC 145, la masse
à transporter est de 3585 kg. Cette charge va
influer sur les dimensions de l’appareil et sur
la puissance à fournir par les turboréacteurs.
Le déplacement des hélicoptères est assuré
par un rotor principal permettant la sustentation
et la translation de l’appareil. Un rotor
arrière permet de compenser le couple de réaction engendré par le rotor principal et
de contrôler les mouvements de lacet de l’appareil. La puissance est délivrée par deux
turboréacteurs (certains hélicoptères ne sont équipés que d’un turboréacteur). Ces
turboréacteurs entraînent en rotation une boîte de transmission principale.
La Boite de Transmission Principale (BTP) permet de distribuer la puissance au
rotor principal, au rotor de queue ainsi qu’à différents accessoires (alternateur, pompe
hydraulique, etc.).



Afin d’évaluer la qualité de la BTP, un banc d’essai permet de la
solliciter et de recréer les conditions de vol. Le diagramme des exigences partiel des exigences est donné figure suivante.


\begin{center}
\includegraphics[width=\linewidth]{images/fig_02}
\end{center}

Le banc d’essai se présente sous la forme d’un châssis permettant d’assurer la liaison avec la BTP. Il est équipé d’un moteur à courant continu piloté par un variateur lui-même alimenté par un transformateur. Ce moteur
entraîne une succession de réducteurs entraînant eux-mêmes deux arbres reliés aux deux
entrées de la BTP. La BTP agit alors sur le rotor principal de l’hélicoptère. Le respect des
caractéristiques des cycles d’essais est assuré par un asservissement en vitesse et en couple.
Des vérins (intégrés au système de simulation des efforts) permettent d’appliquer les efforts simulant la portance de l’appareil en sortie de la BTP.

\begin{obj}
L’objectif de cette étude est de dimensionner l’arbre faisant la jonction entre la
BTP et le banc d’essai.
\end{obj}

\begin{center}
\includegraphics[width=.8\linewidth]{images/fig_03}
\end{center}

\normalsize 




\subsection*{Modélisation de la structure}

Dans le cadre d’un essai de la BTP, les pales ne sont pas utilisées. Il est donc nécessaire
de concevoir un arbre de sortie qui doit faire office de rotor principal. Cet arbre de sortie
devra supporter des efforts équivalents à ceux engendrés par les efforts aérodynamiques.
L’accouplement avec le reste du banc d’essai doit permettre de fournir un couple
résistant. Par ailleurs, des vérins permettent de générer l’équivalent de
l’effort de portance. Une modélisation de l’arbre de sortie de la BTP est présentée sur la
figure  suivante.

\begin{center}
\includegraphics[width=\linewidth]{images/fig_04}
\end{center}

On étudie la phase de vie où l’hélicoptère passe d’une condition de vol stationnaire
à un déplacement. Cette configuration du banc d’essai se traduit par la modélisation
donnée figure suivante. Les efforts sont les suivants en $A$, $B$ et $C$ dans le repère $\repere{O}{x_S}{y_s}{z_s}$.
$
\torseurstat{T}{\text{Ext}_A}{1} = \torseurcol{0}{F_y}{0}{0}{0}{0}{A,\mathcal{R}_S}, 
\torseurstat{T}{\text{Ext}_B}{1} = \torseurcol{-F_x}{-F_y}{0}{0}{0}{0}{A,\mathcal{R}_S}, 
\torseurstat{T}{\text{Ext}_C}{1} = \torseurcol{F_x}{0}{0}{C_t}{0}{0}{A,\mathcal{R}_S}
$.

\begin{center}
\includegraphics[width=\linewidth]{images/fig_05}
\end{center}

\begin{multicols}{2}
\begin{itemize}
\item $OA = \ell_1 = \SI{100}{mm}$;
\item $OB = \ell_2 = \SI{200}{mm}$;
\item $OC = \ell_3 = \SI{300}{mm}$;
\item $F_x = \SI{80000}{N}$; 
\item $F_y = \SI{120000}{N}$;
\item $Ct = \SI{4100}{Nm}$;
\item $R_e = \SI{600}{Mpa}$ 
\item limite d’élasticité au glissement : $R_{eg} = 0,5R_e$.
\end{itemize}
\end{multicols}


\subsection{Évaluer les contraintes dans l’arbre}

\subparagraph{}\textit{Après avoir identifié les différents tronçons à étudier, déterminer le torseur de
cohésion dans l’arbre de sortie de la BTP.}

\subparagraph{}\textit{Tracer les diagrammes des sollicitations associés à chacune des composantes du
torseur de cohésion.}

\subparagraph{}\textit{En ne tenant compte que des sollicitation donnant lieu à une contrainte normale,
donner, dans chacun des tronçons :
\begin{itemize}
\item l’expression vectorielle de contrainte normale;
\item l’allure du champ des contraintes;
\item le diamètre minimal de l’arbre permettant de rester dans le domaine élastique.
\end{itemize}}

\subparagraph{}\textit{}

\subparagraph{}\textit{}

\subparagraph{}\textit{}
\subparagraph{}\textit{}

\subparagraph{}\textit{}

%\subsection*{Flexion de l’arbre intermédiaire 2}
%\begin{hypo}
%Dans cette étude, le candidat ne prendra en compte que le moment de flexion $Mf_y$.
%\end{hypo}

\end{multicols}



\newpage
\begin{center}
\includegraphics[width=\linewidth]{images/cor_01}
\end{center}

\begin{center}
\includegraphics[width=\linewidth]{images/cor_02}
\end{center}

\end{document}

\subparagraph{}\textit{}
\ifprof
\begin{corrige}~\\
\end{corrige}
\else
\fi




\subparagraph{}\textit{}
\ifprof
\begin{corrige}~\\
\end{corrige}
\else
\fi

\subparagraph{}\textit{}
\ifprof
\begin{corrige}~\\
\end{corrige}
\else
\fi

\subparagraph{}\textit{}
\ifprof
\begin{corrige}~\\
\end{corrige}
\else
\fi

\subparagraph{}\textit{}
\ifprof
\begin{corrige}~\\
\end{corrige}
\else
\fi

\subparagraph{}\textit{}
\ifprof
\begin{corrige}~\\
\end{corrige}
\else
\fi

\subparagraph{}\textit{}
\ifprof
\begin{corrige}~\\
\end{corrige}
\else
\fi
\begin{center}
%\includegraphics[width=\linewidth]{images/fig_05}
\end{center}
